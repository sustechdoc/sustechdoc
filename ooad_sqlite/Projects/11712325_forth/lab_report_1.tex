
%%%%%%%%%%%%%%%%%%%%%%%%%%%%%%%%%%%%%%%%%
% University/School Laboratory Report
% LaTeX Template
% Version 3.1 (25/3/14)
%
% This template has been downloaded from:
% http://www.LaTeXTemplates.com
%
% Original author:
% Linux and Unix Users Group at Virginia Tech Wiki 
% (https://vtluug.org/wiki/Example_LaTeX_chem_lab_report)
%
% License:
% CC BY-NC-SA 3.0 (http://creativecommons.org/licenses/by-nc-sa/3.0/)
%
%%%%%%%%%%%%%%%%%%%%%%%%%%%%%%%%%%%%%%%%%

%----------------------------------------------------------------------------------------
%	PACKAGES AND DOCUMENT CONFIGURATIONS
%----------------------------------------------------------------------------------------

\documentclass{article} % if use ctexart, "reference" will be change to Chinese, so we use CJKutf8
\usepackage{multirow}
\usepackage{CJKutf8}
\usepackage{tabu}
%\usepackage{listings}
\usepackage{xcolor}
\usepackage{minted}
\usepackage{cite}
\usepackage{hyperref} % add ref to the [toc]
\usepackage{graphicx} % Required for the inclusion of images
\usepackage{amsmath} % Required for some math elements 
\usepackage{booktabs}
%\setlength\parindent{8pt} % Removes all indentation from paragraphs

\renewcommand{\labelenumi}{\alph{enumi}.} % Make numbering in the enumerate environment by letter rather than number (e.g. section 6)

%\usepackage{times} % Uncomment to use the Times New Roman font

\hypersetup{
	colorlinks=true,
	citecolor=black,
	urlcolor=black,
	linkcolor=black,
	filecolor=black,	
}

%----------------------------------------------------------------------------------------
%	DOCUMENT INFORMATION
%----------------------------------------------------------------------------------------

\title{Report for NCS Parameter Tuning} % Title

\author{
	Zhang Yuefei\\
	11713021
} % Author name

\date{\today} % Date for the report

\begin{document}


\maketitle % Insert the title, author and date
\begin{figure}[h]
	\begin{center}
		\includegraphics[scale=0.5]{SUSTech_University_Logo.png} % Include the logo
	\end{center}
\end{figure}
\clearpage	% add a new page



\tableofcontents	% [toc]
\clearpage
% If you wish to include an abstract, uncomment the lines below
% \begin{abstract}
% Abstract text
% \end{abstract}

%----------------------------------------------------------------------------------------
%	SECTION 1
%----------------------------------------------------------------------------------------

\section{Algorithm Description}
\subsection{Main idea of NCS}
When a team of person deal with a complex task, members of the team tend to cooperate by handling different parts of the task and avoid the same part being done by multiple person. This idea will improve the work efficiency and also inspiring NCS\cite{ncs}, a new population-based search method, which run multiple search processes in parallel and share information between each process to promote a process to search the regions that have not covered before.
\subsection{Application of NCS}
NCS could be use on synthesis of unequally spaced antenna arrays, and other complex (typically non-convex) optimization problems which are ubiquitous in communications and big data analytics\cite{ncs}.
\subsection{Main idea of OLMP}
Because the performance measure of the thresholds is usually a discontinuous and non-differentiable function, so some off-the-shelf methods will not work.
The idea of OLMP is to transform the threshold tuning problem in LMP method into a constrained optimization problem, and use powerful derivative-free optimization algorithms to solve it\cite{olmp}, which not require the problem to be either continuous or differentiable.
\subsection{Application of OLMP}
OLMP solve the main difficulty in applying LMP, which is the threshold tuning problem by automatically adjust threshold in LMP. This facilitate non-professional user to use LMP.
\section{Parameter Description}
%\begin{tabular}{|l|l|l|l|l}}
%	\hline
%	\multirow{2}*{parameters}&\multicolumn{3}{l}{final values and results}&\multirow{2}*{description}\\ \hline
%	&F6&F12&OLMP
%%	\multicolumn{cols}{pos}{text}
%\end{tabular}
\newpage
% Please add the following required packages to your document preamble:
% Please add the following required packages to your document preamble:
% \usepackage{multirow}
% Please add the following required packages to your document preamble:
% \usepackage{multirow}
\begin{table}[]
	\begin{tabular}{lllll}
		\hline
		\multicolumn{1}{|l|}{\multirow{2}{*}{Parameters}} & \multicolumn{3}{l|}{Final values and results} & \multicolumn{1}{l|}{Description} \\ \cline{2-5} 
		\multicolumn{1}{|l|}{} & \multicolumn{1}{l|}{F6} & \multicolumn{1}{l|}{F12} & \multicolumn{1}{l|}{OLMP} & \multicolumn{1}{l|}{} \\ \hline
		\multicolumn{1}{|l|}{} & \multicolumn{1}{l|}{} & \multicolumn{1}{l|}{} & \multicolumn{1}{l|}{} & \multicolumn{1}{l|}{\begin{tabular}[c]{@{}l@{}}OLMP solve the main difficulty in applying LMP, \\ which is the threshold tuning problem by \\ automatically adjust threshold in LMP. \\ This facilitate non-profes\end{tabular}} \\ \hline
		&  &  &  &  \\ \hline
	\end{tabular}
\end{table}



\begin{tabu} to \hsize {|X[4.5]|X[7.3]|X[7.3]|X[7.3]|}
	\hline
	Parameters & F6 & F12 & OLMP \\ \hline
	lambda & 1.0634784249490903 & 1.0272965968507801&1 \\ \hline
	r & 0.846510390324858401 & 0.5457737248418668&2\\ \hline
	epoch & 122 & 2845 &1 \\\hline
	n&1& 2&50\\\hline
	Final Result&390.000063739967&-459.9993145484682&0.98856381981171\\\hline
	Running Time&47.38&32.16&60.29536557197571\\\hline
	
\end{tabu}


\section{Tuning Procedure}

\newpage
%To determine the atomic weight of magnesium via its reaction with oxygen and to study the stoichiometry of the reaction (as defined in \ref{definitions}):

%\begin{center}\ce{2 Mg + O2 -> 2 MgO}\end{center}

% If you have more than one objective, uncomment the below:
%\begin{description}
%\item[First Objective] \hfill \\
%Objective 1 text
%\item[Second Objective] \hfill \\
%Objective 2 text
%\end{description}
%\subsection{Algorithm}
%This project use Minimax Algorithm and Alpha-Beta Pruning.\cite{2005_gomoku}


%----------------------------------------------------------------------------------------
%	BIBLIOGRAPHY
%----------------------------------------------------------------------------------------
\addcontentsline{toc}{section}{Reference}
\begin{CJK}{UTF8}{gbsn} % Chinese reference support
	\bibliographystyle{IEEEtran}
	\bibliography{sample}
\end{CJK}


%----------------------------------------------------------------------------------------


\end{document}